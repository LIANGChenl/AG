\subsection{Schemes}
\begin{exe}
	\label{2.2.1}
	Let $A$ be a ring, let $X= \mathrm{Spec}\, A$, let $f \in A$ and let $D(f) \subseteq X$ be the open complement of $V((f))$. Show that the locally ringed space $\left(D(f),\mathcal{O}_{X}|_{D(f)}\right)$ is isomorphic to $\mathrm{Spec}\, A_{f}$.
\end{exe}
\begin{proof}
	Let $Y$ denote the affine scheme $\operatorname{Spec} A_{f} .$ Then we canonically have a topological open immersion $i: Y \rightarrow X$ whose image is $D(f) .$ Let $D(h)$ be a principal open subset of $X$ contained in $D(f)$. Let $\bar{h}$ be the image of $h$ in $A_{f} .$ We canonically have $$\mathcal{O}_{X}(D(h))=A_{h} \simeq\left(A_{f}\right)_{\bar{h}}=\mathcal{O}_{Y}(D(\bar{h}))=i_{*} \mathcal{O}_{Y}(D(h)) .$$ As the $D(h)$ form a base of open subsets on $D(f)$, this shows that $i$ induces an isomorphism from $\left(Y, \mathcal{O}_{Y}\right)$ onto $\left(D(f),\mathcal{O}_{X}|_{D(f)}\right)$.
\end{proof}
\begin{exe}
	\label{2.2.2}
	Let $\left(X, \mathcal{O}_{X}\right)$ be a scheme, and let $U \subseteq X$ be any open subset. Show that $\left(U,\left.\mathcal{O}_{X}\right|_{U}\right)$ is a scheme. We call this the \emph{induced scheme structure} on the open set $U$, and we refer to $(U,\mathcal{O}_X|_U)$ as an \emph{open subscheme} of $X$.
\end{exe}

\begin{proof}
	Let $\mathrm{Spec}\, A_{i}$ be an affine open cover for $X$. The intersection of each $\mathrm{Spec}\, A_{i}$ with $U$ is an open subset of $\mathrm{Spec}\, A_{i}$ which is therefore covered by basic open affines $D(f_{i j})$. Hence, we obtain an open affine cover $\operatorname{Spec}\left(A_{i}\right)_{f_{ij}}$ for $U$.
\end{proof}

\begin{exe}[Reduced Schemes]
	\label{2.2.3}
	A scheme $\left(X, \mathcal{O}_{X}\right)$ is \emph{reduced} if for every open set $U \subseteq X$, the ring $\mathcal{O}_{X}| _{U}$ has no nilpotent elements.
	
	(a) Show that $\left(X, \mathcal{O}_{X}\right)$ is reduced if and only if for every $P \in X$, the local ring $\mathcal{O}_{X, P}$ has no nilpotent elements.
	
	(b) Let $\left(X, \mathcal{O}_{X}\right)$ be a scheme. Let $(\mathcal{O}_X)_{\mathrm{red}}$ be the sheaf associated to the presheaf $U\mapsto\mathcal{O}_X(U)_{\mathrm{red}}$, where for any ring $A$, we denote by $A_{\mathrm{red}}$ the quotient of $A$ by its ideal of nilpotent elements. Show that $\left(X,\left(\mathcal{O}_{X}\right)_{\mathrm{red}}\right)$ is a scheme. We call it the \emph{reduced scheme} associated to $X$, and denote it by $X_\mathrm{red}$. Show that there is a morphism of schemes $X_{\mathrm{red}} \rightarrow X$, which is a homeomorphism on the underlying topological spaces.
	
	(c) Let $f: X \rightarrow Y$ be a morphism of schemes, and assume that $X$ is reduced. Show that there is a unique morphism $g: X \rightarrow Y_{\mathrm{red}}$ such that $f$ is obtained by composing $g$ with the natural map $Y_{\mathrm{red}} \rightarrow Y$.
	
\end{exe}

\begin{proof}
	(a) Nilpotent is a local-global property. Let $S(A)=\{\text{minimal prime ideals of }A\}$ and $\mathfrak{N}(A)=\{\text{nilpotent elements of }A\}$. Consider an affine open set $\mathrm{Spec}\,A$,  a result of commutative algebra is $\mathfrak{N}(A)=\bigcap _{\mathfrak{p}\in S(A)} \mathfrak{p} $. Thus
	\begin{align*}
		\mathfrak{N}(A)=0 & \iff \bigcap _{\mathfrak{p}\in S(A)} \mathfrak{p}=0 \\
		& \iff\text{for any } \mathfrak{p},\ \bigcap_{\mathfrak{p}_{i} \subseteq \mathfrak{p}}\mathfrak{p}_{i}=0 \\
		&\iff \mathfrak{N}(A_{\mathfrak{p}})=0.
	\end{align*}
	
	(b)  For a basic open affine $D(f)$ we have $$\mathcal{O}_{\operatorname{Spec}\left(A_{\text {red}}\right)}(D(f)) \simeq(A / \mathfrak{N})_{f} \simeq A_{f} /\left(\mathfrak{N}\left(A_{f}\right)\right).$$  That is, on a basic open affine $U$ we have $\mathcal{O}_{\operatorname{Spec}\left(A_{\text {red}}\right)}|_{U} \simeq \mathcal{O}_{(\mathrm{Spec}\, A)_{\text {red}}}|_{U}$. Since the basic opens cover $X$ this shows that $\operatorname{Spec}A_{\mathrm{r e d}} \simeq\left(X,\left(\mathcal{O}_{X}\right)_{\mathrm{r e d}}\right)$.   Now for a general scheme $X$, a cover of $X$ with open affines $\mathrm{Spec}\, A_{i}$ gives a cover $\operatorname{Spec}\left(A_{i}\right)_{\text {red}}$ for $\left(X,\left(\mathcal{O}_{X}\right)_{\text {red}}\right)$. Hence, the latter is a scheme.
	
	
	
	It suffices to check the homeomorphism locally, this follows from the fact that $\mathfrak{N}$ is the intersection of all minimal prime ideals, so that we lose nothing topologically.  
	
	
	(c) We need to prove that any morphism from a reduced scheme to $Y$ factors through $Y_{\mathrm{r e d}}$. We just need to consider the local affine case,  and to patch them all together. Let $V_{i}=$ $\mathrm{Spec}\, B_{i}$ be an open affine cover for $Y$, and let $U_{i j}=\mathrm{Spec}\,A_{i j}$ be an open affine cover of $f^{-1} (V_{i})$. As in the previous part $V_{i}^{\text {red}}=\operatorname{Spec} B_{i}^{\text {red}}$ is an open affine cover for $Y_{\text {red}}$ and the morphism $Y_{\text {red}} \rightarrow Y$ is induced by the ring homomorphisms $B_{i} \rightarrow B_{i}^{\mathrm{r e d}}$. Now since each $A_{i j}$ is reduced, $\mathfrak{N}\left(B_{i}\right)$ is in the kernel of each of the ring homomorphisms $B_{i} \rightarrow A_{i j}$ and so these factor uniquely as $B_{i} \rightarrow B_{i}^{\text {red}} \rightarrow A_{i j} .$ So the morphisms $U_{i j} \rightarrow V_{i}$ factor uniquely as $U_{i j} \rightarrow V_{i}^{\text {red}} \rightarrow V_{i}$. The same is true of each intersection of $U_{i j}$'s and so this gives rise to a unique factorization $f^{-1} (V_{i}) \rightarrow V_{i}^{\text {red}} \rightarrow V_{i}$.
\end{proof}

\begin{exe}
	\label{2.2.4}
	Let $A$ be a ring and let $(X, \mathcal{O}_{X}) $ be a scheme. Given a morphism $f: X \rightarrow$ $\mathrm{Spec}\, A$, we have an associated map on sheaves $f^{\#}: \mathcal{O}_{\mathrm{Spec}\, A} \rightarrow f_{*} \mathcal{O}_{X}$. Taking global sections we obtain a homomorphism $A \rightarrow \Gamma\left(X, \mathcal{O}_{X}\right) .$ Thus there is a natural map
	\begin{equation*}
		\begin{tikzcd}
			{\alpha: \operatorname{Hom}_{\mathfrak{S c h}}(X, \operatorname{Spec} A)} \arrow[r] & {\operatorname{Hom}_{\mathfrak{R i n g}}\left(A, \Gamma\left(X, \mathcal{O}_{X}\right)\right)}
		\end{tikzcd}.
	\end{equation*}
	Show that $\alpha$ is bijective \textup{(cf. (Proposition 1.3.5) for an analogous statement about varieties)}.
\end{exe}
\begin{proof}[Proof $^\dag$]
	The proposition says that $\operatorname{Spec}(-):\mathfrak{R i n g} \rightarrow \mathfrak{Sch} $ is a right adjoint to $\Gamma\left(-, \mathcal{O}_{-}\right): \mathfrak{S c h} \rightarrow \mathfrak{R i n g}$.  Let F be $\Gamma$, G be Spec. That's we need to show there exists two natural transformation:
	\begin{equation*}
		\begin{tikzcd}
			\eta: \mathrm{i d}_{\mathfrak{S c h}} \arrow[r] & \operatorname{Spec} \circ \Gamma\quad \text{and}\quad \varepsilon:\Gamma \circ \mathrm{Spec} \arrow[r] & \mathrm{i d}_{\mathfrak{Rings}}
		\end{tikzcd}
	\end{equation*}
	such that
	$F \xrightarrow{F \eta} F G F \xrightarrow{\varepsilon F} F$ and $G \xrightarrow{\eta G}G F G \xrightarrow{G \varepsilon} G$ equal the identity on F and G respectively. It's straightforward to prove $F \xrightarrow{F \eta} F G F \xrightarrow{\varepsilon F} F$ and the other. The obvious choice for $\varepsilon$ is the isomorphism of Proposition $2.2.2\,(\mathrm{c})$.  For a scheme $X$ we define the natural transformation $\eta$ as follows. Let $U_{i}=\operatorname{Spec} A_{i}$ be an affine cover of the scheme. Each restriction $\Gamma X \rightarrow A_{i}$ gives a morphism $\operatorname{Spec} A_{i} \rightarrow \operatorname{Spec} \Gamma X$ and since the restriction $\Gamma X \rightarrow \mathcal{O}_{X}\left(U_{i}\right) \rightarrow \mathcal{O}_{X}\left(U_{i j}\right)$ is the
	same as $\Gamma X \rightarrow \mathcal{O}_{X}\left(U_{j}\right) \rightarrow \mathcal{O}_{X}\left(U_{i j}\right)$ these morphisms glue to give a morphism $X \rightarrow \operatorname{Spec} \Gamma X$. Given any scheme $X$, we define a morphism $X \rightarrow \operatorname{Spec} \Gamma X$ as follows: for any open affine set $U$, the open immersion $U\hookrightarrow X$ induces a restriction $\mathcal{O}_{X}(X)\rightarrow \mathcal{O}_{X}(U)$, and take the Spec we get the map from $U$ to Spec($\mathcal{O}_{X}(X)$), and we patch the local morphisms all together.
\end{proof}
\begin{exe}
	\label{2.2.5}
	Describe $\mathrm{Spec}\,\mathbb{Z}$, and show that it is a final object for the category of schemes.
\end{exe}

\begin{proof}
	$\mathrm{Spec}\,\mathbb{Z}$ = $\{0\} \cup \{ (p)\,|\,p\ \text{is prime in}\ \mathbb{Z}\}$. The closed sets are all finite sets consist of ideals generated by prime numbers. It is also easy to show that all open subsets of $\mathrm{Spec}\,\mathbb{Z}$ are principle, which is also true for the spectrum of the integral closure of $\mathbb{Z}$ in any number field (see \cite[Ch. 2, Ex. 3.19, P. 58]{LIU}).
	
	Sine $\alpha : \mathrm{Hom}_{\mathfrak{Sch}}(X, \mathrm{Spec}\,\mathbb{Z}) \to \mathrm{Hom}_{\mathfrak{Ring}}(\mathbb{Z}, \Gamma(X, \mathcal{O}_{X}))$ is a bijection by Ex. \ref{2.2.4}. With the fact that there is a unique homomorphism $\mathbb{Z}\to A$, by sending $n$ to $n\cdot1_{A}$ for any unitary commutative ring $A$, we can conclude that there exists a unique morphism $X \to \mathrm{Spec}\,\mathbb{Z}$.
\end{proof}

\begin{exe}
	\label{2.2.6}
	Describe the spectrum of the zero ring, and show that it is an initial object for the category of schemes.
\end{exe}

\begin{proof}
	$\mathrm{Spec}\,0 = \varnothing$. For any scheme $X$, $\varnothing \to X$ is clearly a trivial map, then $\mathrm{Spec}\,0$ is an initial object in the category of schemes.
\end{proof}
\begin{exe}
	\label{2.2.7}
	Let $X$ be a scheme. For any $x\in X$, let $\mathcal{O}_x$ be the local ring at $x$, and $\mathfrak{m}_x$ its maximal ideal. We define the \emph{residue field} of $x$ on $X$ to be the field $k(x)=\mathcal{O}_x/\mathfrak{m}_x$. Now let $K$ be any field. Show that to give a morphism of $\mathrm{Spec}\,K$ to $X$ it is equivalent to give a point $x$ and an inclusion map $k(x)\to K$.
\end{exe}
\begin{proof}
	We want to show that there are bijections
	\begin{equation*}
		\begin{tikzcd}
			{\mathrm{Hom}_{\mathfrak{Sch}}\left(\mathrm{Spec}\,K,X\right)} \arrow[r, "\alpha", shift left] & {\{(x,k(x)\hookrightarrow K)\,|\,x\in X,\ \mathrm{Hom}(k(x),K)\neq\varnothing\}} \arrow[l, "\beta", shift left]
		\end{tikzcd}.
	\end{equation*}
	Firstly, we can see that$$\mathrm{Spec}\,K=\{\eta\}\quad\text{and}\quad\mathcal{O}_{\mathrm{Spec}\,K,\eta}=K.$$
	
	(1) Let $(f,f^\#):\mathrm{Spec}\,K\to X$ and $x=f(\eta)$. Then we have a local homomorphism $f_x^\#:\mathcal{O}_x\to K$ which induces an inclusion $\bar{f}:k(x)\to K$. Thus define $\alpha(f,f^\#)=(x,\bar{f})$.
	
	(2) Given $x\in X$ and an inclusion $i:k(x)\to K$, we can define $f:\mathrm{Spec}\,K\to X$ by sending $\eta$ to $x$ and
	\begin{equation*}
		\begin{tikzcd}
			p:\mathcal{O}_x \arrow[r] & k(x) \arrow[r, "i"] & {K,\quad a} \arrow[r, maps to] & a+\mathfrak{m} \arrow[r, maps to] & i(a+\mathfrak{m}_x)
		\end{tikzcd}.
	\end{equation*}
	With the fact that
	\begin{equation*}
		f_*\mathcal{O}_{\mathrm{Spec}\,K}(U)=\left\{\begin{matrix}
			K,\ &\text{if }x\in U\\
			0,\ &\text{otherwise},
		\end{matrix}
		\right.
	\end{equation*}
	we can define $\beta(x,i)=f^\#:\mathcal{O}_X\to f_*\mathcal{O}_{\mathrm{Spec}\,K}$ as: if $x\notin U$, $f^\#(U)$ is zero map; if $x\in U$,
	\begin{equation*}
		\begin{tikzcd}
			f^\#(U):\mathcal{O}_X(U) \arrow[r] & \mathcal{O}_x \arrow[r, "p"] & {f_*\mathcal{O}_{\mathrm{Spec}\,K}(U)}
		\end{tikzcd}.
	\end{equation*}
	
	It is easy to vertify that $\alpha$ and $\beta$ are invertible to each other with the fact that $f^\#_x=p$ where $f^\#$ and $p$ are defined in (2).
\end{proof}

\begin{exe}
	\label{2.2.8}
	Let $X$ be a scheme. For any $x\in X$, we define the \emph{Zariski tangent space} $T_x$ to $X$ at $x$ to be the dual of the $k(x)$-vector space $\mathfrak{m}_x/\mathfrak{m}_x^2$. Now assume that $X$ is a scheme over a field $k$, and let $k[\varepsilon]/(\varepsilon^2)$ be the ring of dual numbers over $k$. Show that to give a $k$-morphism of $\mathrm{Spec}\,k[\varepsilon]/(\varepsilon^2)$ to $X$ it is equivalent to give a point $x\in X$, \emph{rational over} $k$ (i.e., such that $k(x)=k$), and an element of $T_x$.
\end{exe}

\begin{proof}
	We want to show that there are bijections
	\begin{equation*}
		\begin{tikzcd}
			{\mathrm{Hom}_k\left(\mathrm{Spec}\,k[\varepsilon]/(\varepsilon^2),X\right)} \arrow[r, "\alpha", shift left] & {\{(x,t)\,|\,x\in X,\ \text{rational over }k ,\ \text{and }t\in T_x \}} \arrow[l, "\beta", shift left]
		\end{tikzcd}.
	\end{equation*}
	Use simple notations $S$ and $k[\varepsilon]$ instead of $\mathrm{Spec}\,k[\varepsilon]/(\varepsilon^2)$ and $k[\varepsilon]/(\varepsilon^2)$ respectively. With easy calulation, we have$$S=\{(\varepsilon)\}\quad\text{and}\quad\mathcal{O}_{S,(\varepsilon)}=k[\varepsilon].$$
	
	(1) Assume $(f,f^\#)\in\mathrm{Hom}_k(S,X)$. Let $x=f((\varepsilon))$. Then we get a local $k$-homomorphism
	\begin{equation*}
		\begin{tikzcd}
			{f^\#_x:\mathcal{O}_{X,x}} \arrow[r] & {\mathcal{O}_{S,(\varepsilon)}=k[\varepsilon]}
		\end{tikzcd}
	\end{equation*}
	which induces an inclusion $k(x)\hookrightarrow k[\varepsilon]$. On the other hand, $k(x)$ is a finite extension over $k$. Thus we can conclude that $k(x)=k$; otherwise suppose that $a\in k(x)-k$, then we can write $a=a_1+a_2\varepsilon$ for some $a_1,a_2\in k$, and hence $\varepsilon\in k$ which is a contradiction.
	
	The local $k$-homomorphism $f_x^\#$ maps the maximal ideal $\mathfrak{m}_x$ to $(\varepsilon)=k\varepsilon$. Since $\varepsilon^2=0$, we have $\mathfrak{m}_x^2\subseteq\ker f^\#_x$. Hence with the natural $k$-isomorphism $k\varepsilon\to k$ sending $a\varepsilon$ to $a$, we can obtain a $k$-homomorphism
	\begin{equation*}
		\begin{tikzcd}
			t:\mathfrak{m}_x/\mathfrak{m}_x^2 \arrow[r] & k\varepsilon \arrow[r] & k
		\end{tikzcd}.
	\end{equation*}
	Define $\alpha(f,f^\#)=(x,t)$.
	
	(2) Given a $k$-rational point $x\in X$ and $t\in T_x$, we can define $f:S\to X$ by sending $(\varepsilon)$ to $x$ and
	\begin{equation*}
		\begin{tikzcd}
			q:\mathfrak{m}_x \arrow[r] & \mathfrak{m}_x/\mathfrak{m}_x^2 \arrow[r, "t"] & k \arrow[r] & k\varepsilon
		\end{tikzcd}.
	\end{equation*}
	With the following exact sequence, where $p$ and $i$ are natural projection and inclusion with $p\circ i=\mathrm{id}_k$, we have $\mathcal{O}_{X,x}\cong k\oplus\mathfrak{m}_x$, applying the splitting lemma.
	\begin{equation*}
		\begin{tikzcd}
			0 \arrow[r] & \mathfrak{m}_x \arrow[r] & {\mathcal{O}_{X,x}} \arrow[r, "p"] & k \arrow[r] \arrow[l, "i", bend left] & 0
		\end{tikzcd}.
	\end{equation*}
	Then we can define
	\begin{equation*}
		\begin{tikzcd}
			{q^{\prime}:\mathcal{O}_{X,x}\simeq k\oplus\mathfrak{m}_x} \arrow[r] & {k[\varepsilon]=\mathcal{O}_{S,(\varepsilon)},\quad(a,r)} \arrow[r] & a+q(r)
		\end{tikzcd}.
	\end{equation*}
	With the fact that
	\begin{equation*}
		f_*\mathcal{O}_S(U)=\left\{\begin{matrix}
			k[\varepsilon],\ &\text{if }x\in U\\
			0,\ &\text{otherwise},
		\end{matrix}
		\right.
	\end{equation*}
	we can define $\beta(x,t)=f^\#:\mathcal{O}_X\to f_*\mathcal{O}_S$ as: if $x\notin U$, $f^\#(U)$ is zero map; if $x\in U$,
	\begin{equation*}
		\begin{tikzcd}
			f^\#(U):\mathcal{O}_X(U) \arrow[r] & {\mathcal{O}_{X,x}} \arrow[r, "q"] & f_*\mathcal{O}_S(U)
		\end{tikzcd}.
	\end{equation*}
	
	It is easy to vertify that $\alpha$ and $\beta$ are invertible to each other with the fact that $f^\#_x=q'$ where $f^\#$ and $q'$ are defined in (2).
\end{proof}

\begin{exe}
	\label{2.2.9}
	If $X$ is a topological space, and $Z$ an irreducible closed subset of $X$, a generic point for $Z$ is a point $\zeta$ such that $Z=\{\zeta\}^{-} .$ If $X$ is a scheme, show that every (nonempty) irreducible closed subset has a unique generic point.
\end{exe}
\begin{proof}
	Uniqueness: Suppose $\zeta_{1}, \zeta_{2}$ are two generic points of $Z$. Then since $\{\zeta_{i}\}^-=Z$, an open set contains $\zeta_{1}$ iff it contains $\zeta_{2}$. Letting $U=\operatorname{Spec} A$ be an affine neighbourhood of $\zeta_{1}$, we identify $\zeta_{i}|_U=\mathfrak{p}_{i} \in \operatorname{Spec} R$. Since $\mathfrak{p}_{2} \in \{\mathfrak{p}_{1}\}^-$, we have $\mathfrak{p}_{2} \supseteq \mathfrak{p}_{1}$ and vice versa, so $\zeta_{1}=\zeta_{2}$.
	
	Existence: Let $X$ be a scheme, $Z \subseteq X$ closed and irreducible. If $U \subseteq Z$ is open and $\zeta \in U$ such that $\{\zeta\}^-=U$, then $\{\zeta\}^-=Z$ in $X$ since $Z$ is irreducible. So we can assume that $X=\operatorname{Spec} A$ is affine and $Z=V(\mathfrak{a})\simeq\operatorname{Spec} A / \mathfrak{a}$ for some ideal $\mathfrak{a} \subseteq A .$ Now we can further assume that $Z=X=\operatorname{Spec} A$ is irreducible. It follows that there can only be one minimal prime ideal whose closure is all of $X$.
\end{proof}

\begin{exe}
	\label{2.2.10}
	Describe $\mathrm{Spec}\, \mathbb{R}[x]$. How does its topological space compare to the set $\mathbb{R} ?$ To $\mathbb{C}$?
\end{exe}

\begin{proof}
	The prime ideals of $\mathbb{R}[x]$ fall into three types:
	(1) The generic point $(0)$, with residue field $\mathbb{R}(x)$.
	(2) Closed points of the form $(x-\alpha)$ with $\alpha \in \mathbb{R}$. The residue field in each case is $\mathbb{R}$.
	(3) Closed points of the form $\left(x^{2}+\alpha x+\beta\right)$, with residue field $\mathbb{C}$.
	As a set, there is a bijection between $\operatorname{Spec} \mathbb{R}[x]$ and the upper complex plane $\mathbb{H}$ by sending $z \in \mathbb{H}$ to $(x-z)(x-\bar{z})$ if $z \notin \mathbb{R}$ and
	to $(x-z)$ if $z \in \mathbb{R}$.
	
	There are quite a lot differences between the topological spaces $\mathrm{Spec}\,\mathbb{R}[x]$ and $\mathbb{R}$ (or $\mathbb{C}$). For instance, (1) $\mathbb{R}$ is Hausdorff, but $\mathrm{Spec}\,\mathbb{R}[x]$ is not; (2) $\mathrm{Spec}\,\mathbb{R}[x]$ is quasi-compact, but $\mathbb{R}$ is not; (3) $\mathrm{Spec}\,\mathbb{R}[x]$ has a generic point, but $\mathbb{R}$ does not;
	(4) The map $\mathbb{R}\to\mathrm{Spec}\,\mathbb{R}[x]$ defined above is continuous, but $\mathrm{Spec}\,\mathbb{R}[x]\to\mathbb{H}\hookrightarrow\mathbb{C}$ is not.
\end{proof}

\begin{exe}
	\label{2.2.11}
	Let $k=\mathbb{F}_{p}$ be the finite field with $p$ elements. Describe $\operatorname{Spec} k[x]$. What are the residue fields of its points? How many points are there with a given residue field?
\end{exe}
\begin{proof}
	
	The space: the generic point and one point for every monic irreducible polynomial.  
	
	The residue field of the generic point is the fractional field,  and is 
	$\mathbb{F}_{p^{n}}$ if the point corresponds to a polynomial of degree n.  
	
	Thus to count the number of points with a given residue field $\mathbb{F}_{p^{n}}$,  it suffices to count the number of monic irreducible polynomials with degree $n$.   Moreover it's equivalent to count the number of elements of $\mathbb{F}_{p^{n}}$ not contained in any subfield (i.e.  the order is $p^{n}$),  since every irreducible polynomial $f(x)$ of degree $n$ gives $n$ elements of $\mathbb{F}_{p^{n}}$ via the isomorphism $\mathbb{F}_{p}[x] /(f(x)) \rightarrow \mathbb{F}_{p^{n}}$ and every element $\alpha$ of $\mathbb{F}_{p^{n}}$ that is not contained in any subfields gives an irreducible polynomial of degree $n$ by taking its minimal polynomial $\prod_{i=0}^{n-1}\left(x-\alpha^{p^{i}}\right), $ and these processes are inverses of each other.  Use the Möbius inverse formula we get the answer: $\frac{1}{n}\sum_{d \mid n} \mu(d) p^{d}$.
\end{proof}
\begin{exe}[Glueing Lemma]
	\label{2.2.12}
	Let $\left\{X_{i}\right\}$ be a family of schemes (possibly infinite). For each $i \neq j$, suppose given an open subset $U_{i j} \subseteq X_{i}$ and let it have the induced scheme structure. Suppose also given for each $i \neq j$ an isomorphism of schemes $\varphi_{i j}: U_{i j} \rightarrow U_{j i}$ such that (1) for each $i, j, \varphi_{j i}=\varphi_{i j}^{-1}$, and (2) for each $i, j, k$, $\varphi_{i j}\left(U_{i j} \cap U_{i k}\right)=U_{j i} \cap U_{j k}$, and $\varphi_{i k}=\varphi_{j k}\circ \varphi_{i j}$ on $U_{i j} \cap U_{i k}$.
	
	Then show that there is a scheme $X$, together with morphisms $\psi_{i}: X_{i} \rightarrow X$ for each $i$, such that (1) $\psi_{i}$ is an isomorphism of $X_{i}$ onto an open subscheme of $X$, (2) the $\psi_{i}\left(X_{i}\right)$ cover $X$, (3) $\psi_{i}\left(U_{i j}\right)=\psi_{i}\left(X_{i}\right) \cap \psi_{j}\left(X_{j}\right)$ and (4) $\psi_{i}=\psi_{j} \circ \varphi_{i j}$ on $U_{i j}$. We say that $X$ is obtained by \emph{gluing} the schemes $X_i$ along the isomorphisms $\varphi_{ij}$. An interesting special case is when the family $X_i$ is arbitrary, but the $U_{ij}$ and $\varphi_{ij}$ are all empty. Then the scheme $X$ is called the \emph{disjoint union} of the $X_i$, and is denoted $\coprod X_i$.
\end{exe}
\begin{proof}
	We can view a scheme as a funtor,  and the exercise just says that it's a sheaf in the Zariski site.  We give a brief construction without too much check: 
	
	$\text {Define a topological space } X \text { as the quotient of } \coprod X_{i} \text { by the }$ equivalence relation $x$ $ \sim y$ if $x=y$, or if there are $i, j$ such that $x \in U_{i j} \subseteq$ $X_{i},\ y \in U_{j i} \subseteq X_{j}$, and $\varphi_{i j} (x)=y .$ We take the quotient topology.  Let $\psi:\coprod X_i\to X$ be the quotient map, and let $\psi_i=\psi|_{X_i}$. Now for each $i$ we have a sheaf $\psi_{*} \mathcal{O}_{X_{i}}$ on the image of $X_{i}$ by pushing forward the structure sheaf of $X_{i}$, and on the intersections, we have the pushforward of the isomorphisms $\varphi_{i j}^{\#}$, and these satisfy the required relation to use Ex. \ref{2.1.22} to glue the sheaves together obtaining a sheaf $\mathcal{O}_{X}$ together with isomorphisms $\psi_{i}^\#:\left.\mathcal{O}_{X}\right|_{\psi X_{i}} \stackrel{\sim}{\rightarrow} \psi_{*} \mathcal{O}_{X_{i}} .$ 
\end{proof}

\begin{exe}
	\label{2.2.13}
	A topological space is \emph{quasi-compact} if every open cover has finite subcover.
	
	(a) Show that a topological space is noetherian if and only if every open subset is quasi-compact.
	
	(b) If $X$ is an affine scheme show that $\mathrm{sp} (X)$ is quasi-compact, but not in general noetherian.
	
	(c) If $A$ is a noetherian ring, show that $\mathrm{sp} (\mathrm{Spec}\,A)$ is a noetherian toplogical space.
	
	(d) Give an example to show that $\mathrm{sp}(\mathrm{Spec}\,A)$ can be noetherian even when $A$ is not.
\end{exe}
\begin{proof}
	(a) one direction is trivial.  The other direction follows from: Let $U$ be an open subset and $\left\{U_{i}\right\}$ a cover of $U$.  Define an increasing sequence of open subsets by $V_{0}=\varnothing$ and $V_{i+1}=V_{i} \cup U_{i}$ where $U_{i}$ is an element of the cover not contained in $V_{i} .$ If we can always find such a $U_{i}$ then we obtain a strictly increasing sequence of open subsets of $X$, which contradicts $X$ being noetherian. Hence, there is some $n$ for which $\bigcup_{i=1}^{n} U_{i}=U$ and therefore $\left\{U_{i}\right\}$ has a finite subcover.
	
	(b) Let $\left\{U_{i}\right\}$ be an open cover for $\mathrm{sp}(X)$. The complements of $U_{i}$ are closed and therefore determined by ideals $I_{i}$ in $A=\Gamma\left(\mathcal{O}_{X}, X\right) .$ Since $\bigcup U_{i}=X$ the $I_{i}$ generate the unit ideal and hence $1=\sum_{j=1}^{n} f_{j} g_{i_{j}}$ for some $f_{j}$ where $g_{i_{j}} \in I_{i_{j}} .$ Then $\left\{I_{i_{1}}, \ldots, I_{i_{n}}\right\}$ also generate the unit ideal and therefore we have a finite subcover $\left\{U_{i_{1}}, \ldots, U_{i_{n}}\right\}$.
	
	An example of a non noetherian affine scheme is $\operatorname{Spec} k\left[x_{1}, x_{2}, \ldots\right]$ which has a decreasing chain of closed subsets $V\left(x_{1}\right) \supseteq V\left(x_{1}, x_{2}\right) \supseteq V\left(x_{1}, x_{2}, x_{3}\right) \supseteq\dots$
	
	(c) A decreasing sequence of close subsets $Z_{1} \supseteq Z_{2} \supseteq \ldots$ corresponds to an increasing sequence $I_{1} \subseteq I_{2} \subseteq \ldots$ of ideals of $A .$ Since $A$ is noetherian this stabilizes at some point and therefore, so does the sequence of closed subsets.
	
	
	(d) Consider the ring $A=k\left[x_{1}, x_{2}, \ldots\right] /\left(x_{1}^{2}, x_{2}^{2}, \ldots\right)$. The $\mathrm{Spec}\,A$ consists of only one point, since $\mathrm{Rad}(0)=(x_1,x_2,\dots)$ is a maximal ideal. But it's not a Noetherian ring as $(x_1)\subseteq(x_1,x_2)\subseteq\dots$ in $A$.
\end{proof}
\begin{exe}
	\label{2.2.14}
	(a) Let $S$ be a graded ring. Show that $\mathrm{Proj}\, S = \varnothing$ iff every element of $S_{+}$ is nilpotent.
	
	(b) Let $\varphi  : S \to T$ be a graded homomorphism of graded rings (preserving degrees). Let $U = \{ p \in \mathrm{Proj}\, T\, |\, \varphi(S_{+}) \not\subseteq \mathfrak{p} \}$. Show that $U$ is an open subset of $\mathrm{Proj}\, T$, and show that $\varphi$ determines a natural morphism $f : U \to \mathrm{Proj}\, S$.
	
	(c) The morphism $f$ can be a isomorphism even when $\varphi$ is not. For example, suppose that $\varphi_{d} : S_{d} \to T_{d}$ is an isomorphism for all $d \geq d_{0}$, where $d_{0}$ is an integer. Then show that $U = \mathrm{Proj}\, T$ and the morphism $f : \mathrm{Proj}\, T \to \mathrm{Proj}\, S$ is an isomorphism.
	
	(d) Let $V$ be a projective variety with homogeneous coordinate ring $S$. Show that $t(V) \simeq \mathrm{Proj}\, S$.
\end{exe}

\begin{proof}
	(a) Assume that $\mathrm{Proj}\, S = \varnothing$. Then any homogeneous prime ideal of $S$ contains $S_+$. Thus with the property that $\mathrm{Rad}(0)$ is equal to the intersection of all homogeneous prime ideals of $S$, we can conclude that $S_+\subseteq\mathrm{Rad}(0)$.
	
	Suppose that $S_{+}\subseteq\mathrm{Rad}(0)$. Let $\mathfrak{p}$ be a homogeneous prime ideal of $S$. It is clear that $\mathfrak{p}\supseteq\mathrm{Rad}(0)$ and hence $\mathfrak{p}\supseteq S_{+}$. Thus $\mathrm{Proj}\, S = \varnothing$.
	
	(b) Consider the set $\mathrm{Proj}\, T-U = \{\mathfrak{p}\in\mathrm{Proj}\, T\,|\, \varphi(S_{+})\subseteq \mathfrak{p}\}$. For any $f \in S$, we can rewrite $f = f_{1}+f_{2}+\dots+f_{n}$ with $f_{i}$ homogeneous, so $\varphi(f)=\varphi(f_{1})+\dots+\varphi(f_{n})$. Then the ideal generated by $\varphi(S_{+})$ denoted by $I$ is generated by the images of homogeneous elements of $S$ which are homogeneous elements in $T$. So $\mathrm{Proj}\, T-U=V(I)$ is closed, which means $U$ is open.
	
	For the map $f : U \to \mathrm{Proj}\, S$, with $\mathfrak{p} \mapsto \varphi^{-1}(\mathfrak{p})$. Since $\varphi(S_{+}) \not\subseteq \mathfrak{p}$ implies $S_{+} \not\subseteq \varphi^{-1}(\mathfrak{p})$. Then $f$ is well-defined. Define a local homomophism $\varphi_{(\mathfrak{p})}: S_{(\varphi^{-1}(\mathfrak{p}))} \to T_{(\mathfrak{p})}$, then $\varphi_{(\mathfrak{p})}$ induces a morphism of sheaves $f^{\#} : \mathcal{O}_{\mathrm{Proj}\, S} \to f_{*}U$. So $f$ is a morphism of schemes.
	
	(c) Let $\mathfrak{p} \in \mathrm{Proj}\, T$, and assume $\varphi(S_{+}) \subseteq \mathfrak{p}$. Take $x \in T$ homogeneous with $\deg x = \alpha\ \textgreater 0$. Then we can choose $n\in\mathbb{Z}$ such that $n\alpha \geq d_{0}$. So $x^{n} \in T_{n\alpha} = \varphi(S_{n\alpha}) \subseteq \mathfrak{p}$, which implies $x \in \mathfrak{p}$. Since $\alpha$ is any integer, we get $T_{+} \subseteq \mathfrak{p}$, contradicting with $\mathfrak{p} \in \mathrm{Proj}\, T$. Then $\varphi(S_{+}) \not\subseteq \mathfrak{p}$. Hence $U = \mathrm{Proj}\, T$.
	
	\emph{The morphism $f$ is an isomorphism.}
	
	If $d_0\leq0$, $\varphi$ is an isomorphism, in which case the conclusion is trivial. Thus we assume that $d_0\geq1$.
	
	Firstly we show that $\{D^S_+(g)\,|\,\deg g\geq d_0\}$ is an open affine cover of $\mathrm{Proj}\,S$. If not, assume that $\mathfrak{p}\in\mathrm{Proj}\,S$ is not contained in $D^S_+(g)$, i.e. $g\in\mathfrak{p}$, for any $g\in S$ with $\deg g\geq d_0$. Then for any $x\in S_+$, $x^n\in\mathfrak{p}$ since $\deg x^n\geq d_0$ for some $n>0$, and hence $x\in\mathfrak{p}$. Thus $S_+\subseteq\mathfrak{p}$, which is a contradiction. Moreover since $\varphi_d$ is an isomorphism for each $d\geq d_0$, $\{D^T_+(\varphi(g))\,|\,g\in S,\ \deg g\geq d_0\}$ is an open affine cover of $\mathrm{Proj}\,T$.
	
	With thw fact that $D^S_+(g)=\mathrm{Spec}\,S_{(g)}$ and $D^T_+(\varphi(g))=\mathrm{Spec}\,T_{(\varphi(g))}$, by Ex. \ref{2.2.17}\,(a) and Ex. \ref{2.2.18}, it suffices to show that $\varphi_{(g)}:S_{(g)}\to T_{(\varphi(g))}$ is an isomorphism for any $g\in S$ with $\deg g\geq d_0$, which immediately follows the assumption that $\varphi_d$ is an isomorphism for all $d\geq d_0$.
\end{proof}

\begin{exe}
	\label{2.2.15}
	(a) Let $V$ be a variety over the algebraically closed field $k$. Show that a point $P \in t(V)$ is a closed point if and only if its residue field is $k$.
	
	(b) If $f : X \to Y$ is a morphism of schemes over $k$, and if $P \in X$ is a point with residue field $k$, then $f(P) \in Y$ also has residue field $k$.
	
	(c) Now show that if $V,W$ are two varieties over $k$, then the natural map 
	\begin{equation*}
		\begin{tikzcd}
			{\mathrm{Hom}_{\mathfrak{Var}}(V,W)} \arrow[r] & {\mathrm{Hom}_{\mathfrak{Sch}/k}(t(V),t(W))}
		\end{tikzcd}
	\end{equation*}
	is bijective.
\end{exe}

\begin{proof}
	
\end{proof}

\begin{exe}
	\label{2.2.16}
	Let $X$ be a scheme, let $f\in\Gamma(X,\mathcal{O}_X)$, and define $X_f$ to be the subset of points $x\in X$ such that the stalk $f_x$ of $f$ at $x$ is not contained in the maximal ideal $\mathfrak{m}_x$ of the local ring $\mathcal{O}_x$.
	
	(a) If $U=\mathrm{Spec}\,B$ is an open \emph{affine} subscheme of $X$, and if $\bar{f}\in B=\Gamma(U,\mathcal{O}_X|_U)$ is the restriction of $f$, show that $U\cap X_f=D(\bar{f})$. Conclude that $X_f$ is an open subset of $X$.
	
	(b) Assume $X$ is quasi-compact. Let $A=\Gamma(X,\mathcal{O}_X)$, and let $a\in A$ be an element whose restriction to $X_f$ is 0. Show that for some $n>0$, $f^na=0$.
	
	(c) Now assume that $X$ has a finite cover by open affines $U_i$ such that the intersection $U_i\cap U_j$ is quasi-compact. (This hypothesis is satisfied, for example, if $\mathrm{sp}(X)$ is noetherian.) Let $b\in\Gamma(X_f,\mathcal{O}_{X_f})$. Show that for some $n>0$, $f^nb$ is the restriction of an element of $A$.
	
	(d) With the hypothesis of (c), conclude that $\Gamma(X_f,\mathcal{O}_{X_f})\simeq A_f$.
\end{exe}
\begin{proof}
	(a) Assume $x\in D(\bar{f})$. Then $\bar{f}\notin x$ and hence $f_x=\bar{f}_x=\bar{f}\notin xB_x=\mathfrak{m}_x$ viewing $B$ as a subring of $B_x$. Thus $D(\bar{f})\subseteq X_f$. On the Other hand, let $x\in U\cap X_f$. Then $\mathcal{O}_x=B_x$ and $\mathfrak{m}_x=xB_x$. Thus the condition that $f_x\notin\mathfrak{m}_x$ implies that $\bar{f}\notin x$ and hence $x\in D(\bar{f})$. Therefore, $U\cap X_f\subseteq D(\bar{f})$. Then we obtain that $U\cap X_f=D(\bar{f})$.
	
	Let $\{U_i\}_{i\in I}$ be an open affine cover of $X$ and let $f_i=f|_{U_i}$. Then $X_f=\bigcup_{i\in I}(U_i\cap X_f)=\bigcup_{i\in I}D_{U_i}(f_i)$ is open.
	
	(b) Choose a finite open affine cover $\{U_i\}_{i=1}^l$ of $X$ with $U_i=\mathrm{Spec}\,B_i$, and write $f_i=f|_{U_i}$ and $a_i=a|_{U_i}$. By (a), $U_i\cap X_f=D_{U_i}(f_i)$ and hence $\mathcal{O}_X(U_i\cap X_f)=(B_i)_{f_i}$. Since $a|_{X_f}=0$, the image of $a_i$ in $(B_i)_f$ is $a_i|_{U_i\cap X_f}=0$, and then by the definition of localization $f^{n_i}a|_{U_i}=f_i^{n_i}a_i=0$. Let $n=\max\{n_1,\cdots,n_l\}$. We have $f^na|_{U_i}=0$ for all $i=1,\cdots,l$, hence $f^na=0$.
	
	(c) Assume that $U_i=\mathrm{Spec}\,B_i$ and write $b_i=b|_{U_i\cap X_f}$. By (a), $\mathcal{O}_X(U_i\cap X_f)=(B_i)_{f_i}$, and hence we have $f_i^{k_i}b_i\in B_i$ for some $k_i>0$ by the definition of localization. Choose the maximal $k_i$ denoted by $k$. Then $f_i^kb_i\in\mathcal{O}_X(U_i)$ for all $i$. Let $b_{ij}=f_i^kb_i|_{U_i\cap U_j}-f_j^kb_j|_{U_i\cap U_j}$. Then $b_{ij}|_{U_i\cap U_j\cap X_f}=0$. Since $U_i\cap U_j$ is quasi-compact, by (b), $f_{ij}^{m_{ij}}b_{ij}=0$ for some $m_{ij}>0$ where $f_{ij}=f|_{U_i\cap U_j}$. Choose the maximal $m_{ij}$ denoted by $m$. Let $n=m+k$. Then we have that $f_i^nb_i\in B_i=\mathcal{O}_X(U_i)$ and $f_i^nb_i|_{U_i\cap U_j}=f_j^nb_j|_{U_i\cap U_j}$. Therefore there exists $t\in\Gamma(X,\mathcal{O}_X)$ such that $t|_{X_f}=f^nb$.
	
	(d) We will use the notations in (c). By (b) and the definition of localization, it is clear that $\Gamma(X_f,\mathcal{O}_{X_f})\subseteq A_f$. On the other hand, let $f^{-n}a\in A_f$ where $a\in A$ and $n>0$. By (a), we can map $f^{-n}a$ into $\mathcal{O}_X(U_i\cap X_f)$ with image $s_i=f_i^{-n}(a|_{U_i})$ (this follows the universal property of localization). With the fact that $t_i|_{U_i\cap U_j\cap X_f}=f_{ij}^{-n}(a|_{U_i\cap U_j\cap X_f})=t_j|_{U_i\cap U_j\cap X_f}$, we have that $A_f\subseteq\ker\psi=\Gamma(X_f,\mathcal{O}_{X_f})$, where $\psi$ is defined as:
	\begin{equation*}
		\begin{tikzcd}
			& A_f \arrow[rd]                                            &                                                         &                                                     \\
			0 \arrow[r] & {\Gamma(X_f,\mathcal{O}_{X_f})} \arrow[r] \arrow[u, hook] & \bigoplus_i\mathcal{O}_X(U_i\cap X_f) \arrow[r, "\psi"] & {\bigoplus_{i,j}\mathcal{O}_X(U_i\cap U_j\cap X_f)} \\
			& A \arrow[r] \arrow[u]                                     & \bigoplus_i\mathcal{O}_X(U_i) \arrow[u]                 &                                                    
		\end{tikzcd}
	\end{equation*}
	Finally, we get that $\Gamma(X_f,\mathcal{O}_{X_f})=A_f$.
\end{proof}


\begin{exe}[A Criterion for Affineness]
	\label{2.2.17}
	\ 
	
	(a) Let $f: X \rightarrow Y$ be a morphism of schemes, and suppose that $Y$ can be covered by open subsets $U_{i}$, such that for each $i$, the induced map $f^{-1}\left(U_{i}\right) \rightarrow U_{i}$ is an isomorphism. Then $f$ is an isomorphism.
	
	(b) A scheme $X$ is affine if and only if there is a finite set of elements $f_{1}, \ldots, f_{r} \in A=\Gamma\left(X, \mathcal{O}_{X}\right)$ such that the open subsets $X_{f_{i}}$ are affine, and $f_{1}, \ldots, f_{r}$ generate the unit ideal in $A$.
\end{exe}
\begin{proof}
	(a) Topologically, the two spaces are of course isomorphic. And to check the isomorphism of sheaf structure, it suffices to check it locally.
	
	(b) One direction is obvious. Conversely,  Consider the morphism $f: X \rightarrow \mathrm{Spec}\, A$.  Since the $f_{i}$ generate $A$, the principal open sets $D\left(f_{i}\right)=\mathrm{Spec}\, A_{f_{i}}$ cover $\operatorname{Spec} A$. Their pre-images are $X_{f_{i}}$, which by assumption are affine, isomorphic to $\mathrm{Spec}\, A_{i}$. So the morphism restricts to the morphism $\varphi_{i}: \operatorname{Spec} A_{i} \rightarrow \mathrm{Spec}\, A_{f_{i}}$. Now we just need to show that $\varphi_{i}$ is an isomorphism so that the result follows from part a). Equivalently, we need to show that $\varphi_{i}: \Gamma\left(X, \mathcal{O}_{X}\right)_{f_{i}} \rightarrow \Gamma\left(X_{f_{i}}, \mathcal{O}_{X}\right)$
	is an isomorphism for each $i,$  and the result follows from Ex. \ref{2.2.4}.  The condition of Ex. \ref{2.2.16}\,(d) is satisfied if the under space is Noetherian.  
	
	Injectivity:
	
	Let $f_{i}^{-n}a \in A_{f_{i}}$ and suppose that $\varphi_{i}\left(f_{i}^{-n}a\right)=0$. That means that it vanishes in each of the intersection $X_{f_{i}} \cap X_{f_{j}}=\operatorname{Spec}\left(A_{j}\right)_{f_{i}}$.  So for each $j$ there is some $n_{j}$ such that $f_{j}^{n_{j}}a=0$ in $A_{j} .$ Choosing $m$ large enough, the restriction of $f_{i}^{m} a$ to each open set in a cover vanishes. So $f_{i}^{m} a=0$ and in particular, $f_{i}^{-n}a=0$ in $A_{f_{i}}$.
	
	Surjetivity:
	
	Let $a \in A_{i} .$ For each $j \neq i$, we have $\mathcal{O}_{X}\left(X_{f_{i} f_{j}}\right) \simeq\left(A_{j}\right)_{f_{i}}$ so $a|_{X_{f_{i} f_{j}}}$ can be written as $f_{i}^{-n_{j}}b_j$ for some $b_{j} \in A_{j} .$ That is, we have elements $b_{j} \in A_{j}$ whose restrictions to $X_{f_{i} f_{j}}$ is $f_{i}^{n_{j}} a$. Since there are finitely many, we can choose them so that all the $n_{i}$ are the same, say $n .$ Now on the triple intersections $X_{f_{i} f_{j} f_{k}}=\operatorname{Spec}\left(A_{j}\right)_{f_{i} f_{k}}$ $=\operatorname{Spec}\left(A_{k}\right)_{f_{i} f_{j}}$ we have $$b_{j}|_{X_{f_if_jf_k}}-b_{k}|_{X_{f_if_jf_k}}=f_{i}^{n} a-f_{i}^{n} a=0$$ and so we can find some integer $m_{j k}$ such that $$f_{i}^{m_{j k}}\left(b_{j}|_{X_{f_jf_k}}-b_{k}|_{X_{f_jf_k}}\right)=0.$$ Replacing each $m_{j k}$ by a large enough $m$, we have a section $f_{i}^{m} b_{j}$ for each $X_{f_{j}}$ for $j \neq i$ together with a section $f_{i}^{n+m} a$ on $X_{f_{i}}$ and these sections all agree on intersections. This gives us a global section $d$ whose restriction to $X_{f_{i}}$ is $f_{i}^{n+m} a$ and so $f_{i}^{-n-m}d$ gets mapped to $a$ by $\varphi_{i}$.
\end{proof}
\begin{exe}
	\label{2.2.18}
	Let $A, B$ be rings, $X=\mathrm{Spec}\,A$, $Y=\mathrm{Spec}\,B$. Let $\varphi : A\rightarrow B$ be morphisms of rings, $f: Y\rightarrow X $
	be the induced morphism and $f^\#:\ O_X \rightarrow f_*O_Y$ be the map of sheaves.
	
	(a) For $h\in A$, show that $h$ is nilpotent if and only if $D(h)=\varnothing$.
	
	(b) Show $\varphi$ is injective if and only if $f^\#$ is injective. And show furthermore in that case $f$ is \emph{dominant}, i.e. $f(Y)$ is dense in $X$.
	
	(c) Show that if $\varphi$ is surjective, then $f$ is a homeomoephism of $Y$ onto a closed subset of $X$, and $f^\#$ is surjective.
	
	(d) Prove the converse to (c), namely, if $f:Y\to X$ is a homeomorphism onto a closed subset, and $f^\#:\mathcal{O}_X\to f_*\mathcal{O}_Y$ is surjective, then $\varphi$ is surjective.
\end{exe}

\begin{proof}
	(a) $D(h)=\varnothing\iff h\in \mathfrak{p},\ \forall\, \mathfrak{p}\in \mathrm{Spec}\,A\iff h\in \mathrm{Rad}(0)$. 
	
	(b) Firstly, assume $\varphi$ is injective. The injectivity of $f^\#$ follows the flatness of localization.
	
	Secondly assuming $f^\#$ is injective, take the morphism of global sections and then we conclude that $\varphi$ is injective.
	
	Finally assuming that $\varphi$ is injective and $f$ is not dominant, then we have $h\in A$, such that $D(h)\cap Y=\varnothing$, and $D(h)\neq\varnothing$. i.e. for any $\mathfrak{p}\in \mathrm{Spec}\,B$, $h\in\varphi^{-1}(\mathfrak{p})$.
	So $\varphi(h)\in \mathrm{Rad}(0)$. Then there exists $n$, such that $\varphi(h^n)=\varphi(h)^n=0$. Since $\varphi$
	is injective, $h^n=0$. By (a), $D(h)=\varnothing$, which is a contradiction.
	
	(c) Let $\alpha=\ker\varphi$, it is easy to check that $Y$ is homeomorphic to $V(\alpha)$ via $f$ and
	$\varphi_\mathfrak{p}: A_{\varphi^{-1}(\mathfrak{p})}\longrightarrow B_\mathfrak{p}$ is surjective.
	
	(d) Let $\alpha=\ker\varphi$, consider $\phi:A\rightarrow A/\alpha$, then $\varphi$ induces $\varphi':A/\alpha \rightarrow B $,
	which is injective. By (b) and (c), the induces morphism of scheme $f':\mathrm{Spec}\,B\rightarrow \mathrm{Spec}\,A/\alpha$ is dominant, and
	$i:\mathrm{Spec}\,A/\alpha\rightarrow \mathrm{Spec}\,A$ induces a homeomorphism from $\mathrm{Spec}\,A/\alpha$ to $V(\alpha)$. By the assumption, $f=i\circ f'$ is
	a homeomorphism, then $f'$ is injective. Since $f(Y)$ is closed, $f'(Y)$ is closed. Therefore $f'(Y)=\mathrm{Spec}\,A/\alpha$, since $f'$ is dominant.
	Then $f'$ is a homeomorphism. Consider the induced maps on stalks we derive $f'^\#_\mathfrak{p}:(A/\alpha)_{\varphi^{-1}(\mathfrak{p})}\rightarrow B_\mathfrak{p}$ are isomorphisms.
	Consequently, $f': \mathrm{Spec}\,B \rightarrow \mathrm{Spec}\,A/\alpha$ is a isomorphism of schemes, which implies $\varphi'$ is a isomorphism.
\end{proof}

\begin{exe}
	\label{2.2.19}
	Let $A$ be a ring. Show that the following conditions are equivalent:
	
	\textup{(\romannumeral 1)} $\mathrm{Spec}\,A$ is disconnected;
	
	\textup{(\romannumeral2)} there exists nonzero elements $ e_1,e_2\in A$, such that $e_1^2=e_1$, $e_2^2=e_2$, $e_1+e_2=1$ (these elements are called \emph{orthogonal idempotents});
	
	\textup{(\romannumeral3)} $A$ is isomorphic to a direct product $A_1\times A_2$ of two nonzero rings.
\end{exe}

\begin{proof}
	(i)$\implies$(ii). Assume that $\mathrm{Spec}\,A=U_1\cup U_2$, with that $U_1$ and $U_2$ are disjoint open subsets. Then $U_i$ are closed, i.e. there exist ideals $I_i$ of $A$ such that $U_i=V(I_i)$.
	Then we have $I_1+I_2=A$, and $I_1I_2\subseteq \mathfrak{p}$,  for any $\mathfrak{p}\in \mathrm{Spec}\,A$. Then there exists $f\in I_1$, $g\in I_2$, such that $f+g=1$. Since $fg\in \mathrm{Rad}(0)$,
	$(fg)^n=0$ for some $n$. So the fact that $(f+g)^{2n}=1$ implies that $cf^n+dg^n=1$ for some $c,d\in A$ nonzero. Take $e_1=cf^n$, $e_2=dg^n$.
	
	(ii)$\implies$(iii) Trivial.
	
	(iii)$\implies$(i) Take $f=(0,1)$, $g=(1,0)$, then $\mathrm{Spec}\,A= D(f)\cup D(g)$, and $D(f)\cap D(g)=\varnothing$.
\end{proof}